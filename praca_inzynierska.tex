\documentclass{article}

\usepackage{polski}
\usepackage[utf8]{inputenc}

\title{Praca inżynierska}
\date{2017-10-01}
\author{Jędrzej Kozal}

\usepackage{graphicx}

\begin{document}

\begin{titlepage}
	\centering
	\includegraphics[width=0.25\textwidth]{/home/jkozal/Obrazy/logo_pol_wroclaw.png}\par\vspace{1cm}
	{\scshape\LARGE Politechnika Wrocławska \par}
	\vspace{1cm}
	{\scshape\Large Praca inżynierska\par}
	\vspace{1.5cm}
	{\huge\bfseries Biometryczny system kontroli dostępu - przetwarzanie obrazów i przygotowanie danych dla sieci neuronowych \par}
	\vspace{2cm}
	{\Large\itshape Jędrzej Kozal\par}
	\vfill
	promotor\par
	Dr inż.~Piotr \textsc{Ciskowski}

	\vfill

% Bottom of the page
	{\large 2017-10-01\par}
\end{titlepage}


\section{Wstęp}
Celem pracy jest porównanie metod i algorytmów umożliwiających wykrycie twarzy na zdjęciu, oraz wyznaczenie wektora uczącego dla sieci neuronowej na podstawie wyznaczonego zdjęcia twarzy. Projekt ten jest częścią systemu biometrycznej kontroli dostępu. Za implementację sieci neuronowych oraz analizę algorytmów związanych z sieciami odpowiadał Filip Guzy. Za architekturę systemu, oraz komunikację pomiędzy komponentami odpowiadał Michał Leś.
\subsection{Omówienie zagadnienia}

\subsection{Omawiany komponent jako część większego systemu}
\subsubsection{Wykorzystane wzorce projektowe}

W celu ułatwienia pracy współpracownikom w projekcie został utworzony wydzielony łatwo wymienialny komponent. W celu ułatwienia pracy badawczej komponent został podzielony na mniejsze elementy z wykorzystaniem obiektowych wzorców projektowych.
Do reprezentacji całego komponentu wybrano wzorzec fabryka. Zapewniono w ten sposób elastyczność użytkownikom klasy, ponieważ mogą oni zdecydować w jaki sposób współrzędne będące wynikiem działania algorytmów mają być przedstawione. Zapewniono klasę bazową do przechowywania współrzędnych, po której dziedziczą poszczególne reprezentacje, takie jak bezpośrednie przechowywanie danych w pamięci programu, czy zapis na pamięć nieulotną w postaci pliku z rozszerzeniem .npy z biblioteki numpy.
W celu zwiększenia elastyczności w obrębie komponentu do wyboru algorytmów dwukrotnie wykorzystano wzorzec strategia. Stworzono klasy bazowe do reprezentacji podstawowych własności algorytmu, które następnie implementowano w klasach pochodnych. Modyfikacja ta ułatwiła pracę badawczą i usystematyzowała strukturę projektu.
\subsubsection{Wykorzystane zasady i dobre praktyki programowania}

W trakcie realizacji projektu oprócz wykorzystania wzorców projektowych posługiwano się także dobrymi zasadami SOLID oraz clean code. Pozwalają one na zachowanie większego porządku oraz czytelności kodu. Dobra organizacja kodu oraz porządek ułatwiają rozwijanie projektu oraz systematyzują pracę. Nie zdecydowano się na wykorzystanie testów jednostkowych ze względu na dynamiczny charakter projektu. Definiowanie zachowań w testach jednostkowych jest kosztowne, a częste zmiany powodują że praca ta momentami byłaby zbędna.

\subsubsection{Wykorzystane biblioteki, narzędzia i zasoby}
Projekt systemu biometrycznej kontroli dostępu został zrealizowany całkowicie w języku programowania Python. Decyzja ta była podyktowana głównie znaczącą ilością gotowych bibliotek, które znacząco ułatwiają pracę. Wykorzystana została wersja języka 2.7.9, ponieważ biblioteki języka Python często wymagają wersji 2.7 lub wyższej. W omawianym komponencie wykorzystano biblioteki NumPy, SciPy, scikit-learn oraz openCV. 
NumPy to biblioteka zawierająca wiele algorytmów do realizacji obliczeń numerycznych oraz implementację wielu matematycznych narzędzi związanych z algebrą liniową. Implementacja algorytmów macierzowych z NumPy cieszy się dużą popularnością, dlatego warto poświęcić czas na poznanie tej biblioteki, ponieważ stanowi ona fundament wielu innych projektów i bibliotek.
SciPy jest biblioteką wykorzystywaną do obliczeń naukowych i inżynierskich. Zawiera algorytmy umożliwiające przeprowadzane obliczeń w wielu dziedzinach, dlatego również jest wykorzystywana jako baza innych projektów.
Instalacja SciPy jest prerekwizytem do instlacji scikit-learn. Scikit-learn to biblioteka zawierająca wiele algorytmów z dziedziny machine learning oraz pattern recognition. Jest to kluczowa bilioteka w omawianym projekcie, ponieważ dostarcza najistatniejszych narzędzi. Jej zastosowanie znacznie ułatwiło pracę.
OpenCv stanowi poteżny zbiór narzędzi do analizy oraz przetwarzania obrazów. Rozpowszechniana w ramach licencji . Twórcy biblioteki skupiają się na działanie w czasie rzeczywistym i zapewniają wielordzeniowe przetwarzanie. Jest to znaczące ułatwienie biorąc pod uwagę podstawową platformę sprzętową, na której był realizowany projekt. 

Wszystkie wymienione biblioteki są rozpowszechniane w ramach licencji new-BSD lub 3-clause BSD, co oznacza że wolno je wykorzystywać w celach akademickich lub komercyjnych. Zastosowanie bibliotek dostarczających zoptymalizowanych algorytmów znacznie ułatwiło i usystematyzowało pracę nad projektem.

Na początku pracy nad projektem podjęto próbę implementacji własnej wersji algorytmu PCA, aby lepiej zrozumieć i dokładnie prześledzić jego działanie. Algorytm ten działał poprawnie, ale względu na dużą złożoność obliczeniową zdecydowano się na korzystanie z wersji udostępnionej w bibliotece scikit-learn. W trakcie późniejszych prac nad systemem znaleziono informację na temat tego co powodowało tak wysoką złożoność obliczeniową oraz sposób na jej uniknięcie. Zagadnienie to zostało opisane w rozdziale Realizacja algorytmu oraz sztuczka związana z mnożeniem macierzy.

Wykorzystane bazy zdjęć



\section{Implementacja algorytmów – detekcja twarzy}
 
\subsection{Przetwarzanie obrazu po akwizycji} 
 
 
 
 
\section{Implementacja algorytmów – ekstrakcja cech}
 
\subsection{PCA}
\subsubsection{Wstęp matematyczny}

\subsubsection{Realizacja algorytmu oraz sztuczka związana z mnożeniem macierzy} 


\subsubsection{Interpretacja}


\subsubsection{PCA jako samodzielny system rozpoznawania twarzy}
 
 
\section{Zebranie zbioru danych uczących}
\subsection{Metodyka}

Ile obrazów było w zbiorze uczącym dla PCA

Badanie wpływu oświetlenia
Badanie wpływu kątu pod jakim zrobiono zdjęcie
Badanie wpływu okularów?



\subsection{Automatyzacja badań}


\end{document}